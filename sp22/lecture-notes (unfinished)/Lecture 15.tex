\documentclass{article}
\usepackage[margin=1in]{geometry}
\usepackage{amsmath, amssymb, amsthm}
\usepackage{enumitem}


%Vertical Dots in Align Environment
\usepackage{mathtools}

\usepackage{cancel}

%Formatting and Spacing
\setitemize[1]{noitemsep, parsep = 5pt, topsep = 5pt}
\setenumerate[1]{label = (\alph*), parsep = 1pt, topsep = 5pt}
\setlength\parindent{0pt}
\linespread{1.1}

%Cases Environment
\newlist{Cases}{enumerate}{3}
\setlist[Cases]{leftmargin = .25in, label = {Case \arabic*.}, topsep = 0.01in, itemsep = 0.04in, itemindent = .5in, parsep = 0in}

%Custom Title Fields
\newcommand{\lectTitle}{Lecture 14 Notes}
\newcommand{\lectTime}{March 7, 2022}
\newcommand{\lectClass}{Honors Discrete Mathematics}
\newcommand{\lectClassInstructor}{Gerandy Brito}
\newcommand{\lectSection}{Spring 2022}
\newcommand{\lectAuthorName}{Sarthak Mohanty}

%Headers and Footers
\usepackage{fancyhdr}
\usepackage{extramarks}
\pagestyle{fancy}
\lhead{\lectTime}
\chead{\lectClass \ (\lectClassInstructor)}
\rhead{\lectTitle}
\cfoot{\thepage}
\renewcommand\headrulewidth{0.4pt}
\renewcommand\footrulewidth{0.4pt}

\title{
    \vspace{2in}
    \textbf{\lectClass:\\ \lectTitle}\\
    \vspace{0.1in}\large{\textit{\lectClassInstructor\ \lectSection}}
    \vspace{3in}
    \author{\textbf{\lectAuthorName}}
    \date{}
}

\begin{document}

\maketitle
\pagebreak

\section*{Operations on Congruency Classes}
    \textbf{Example. } Find all integer solutions to the equation $$6x + 2y^{2} = 2020$$
    
    \vspace{1.5mm}\textbf{Solution.}
    We will work step by step.
    \begin{enumerate}[label = \arabic*.]
        \item For any $m \in \mathbb{Z}$, we have $6x + 2y^2 \equiv 2020 \pmod{m}$. We will use $m = 3$. We will first show that $y^{2} \equiv 0$ or $1 \pmod{3}$ for any $y \in \mathbb{Z}$.
        
        \vspace{1.5mm}
        For any $y \in \mathbb{Z}$, either $3 \mid y$ or $3 \nmid y$.
        \begin{Cases}
            \item Suppose $3 \mid y$. Then $y \equiv 0 \pmod{3}$, so $y^{2} \equiv 0 \pmod{3}$.
            \item Suppose $3 \nmid y$. Then $y \equiv 1 \text{ or } 2 \pmod{3}$.
            
            \begin{Cases}[label* = \arabic*., itemindent = .6in, topsep = 0.04in]
                \item Suppose $y \equiv 1 \pmod{3}$. Then $y^{2} \equiv 1^2 \equiv 1 \pmod{3}$. 
                \item Suppose $y \equiv 2 \pmod{3}$. Then $y^{2} \equiv 2^{2} \equiv 1 \pmod{3}$.
            \end{Cases}
            In either subcase, $y^{2} \equiv 1 \pmod{3}$.
        \end{Cases}
        In either case, $y^{2} \equiv 0$ or $1 \pmod{3}$.
        
        \item Now we can show that $6x + 2y^2 \equiv 0$ or $2 \pmod{3}$ for all pairs $x, y \in \mathbb{Z}$.
        
        \vspace{1.5mm}
        For any $y \in \mathbb{Z}$, either $y^{2} \equiv 0 \pmod{3}$ or $y^{2} \equiv 1 \pmod{3}$.
        
        \begin{Cases}
            \item Suppose $y^{2} \equiv 0 \pmod{3}$. Then $6x + 2y^{2} \equiv 6x \equiv 0 \pmod{3}$ for all $x \in \mathbb{Z}$.
            \item Suppose $y^{2} \equiv 1 \pmod{3}$. Then $6x + 2y^{2} \equiv 6x + 2 \equiv 2 \pmod{3}$ for all $x \in \mathbb{Z}$.
        \end{Cases}
        In either case, $6x + 2y^2 \equiv 0$ or $2 \pmod{3}$ for all $x \in \mathbb{Z}$.
        \item Finally, let $x, y \in \mathbb{Z}$. Now note that $2020 \equiv 1 \pmod{3}$. But from part (b), $6x + 2y^{2} \not\equiv 1 \pmod{3}$, so $6x + 2y^{2} \not\equiv 2020 \pmod{3}$, so $6x + 2y^{2} \ne 2020$. Therefore there exist no integer solutions to the equation $6x + 2y^{2} = 2020$.
    \end{enumerate}

\subsection*{Divisibility Rules}
    \textbf{Theorem. } A number $n$ is divisible by $3$ iff the sum of its digits (in base 10) is also a multiple of $3$.
    
    \vspace{1.5mm}
    \textbf{Proof. } Let $n = n_{k}n_{k - 1}n_{k - 2}\dots n_{1}n_{0}$, where $n_{i} \in \{0, 1, \dots, 9\}$. Note that 
    \begin{align*}
        10 &\equiv 1 \pmod{3} \\
        10^{2} &\equiv 1 \pmod{3} \\
        &\vdotswithin{ = } \\
        10^{t} &\equiv 1 \pmod{3} \tag{$*$}
    \end{align*}
    Then 
    $$\begin{aligned}[t]
        n &= \cancelto{1}{10^{k}} \cdot n_{k} + \cancelto{1}{10^{k - 1}} \cdot n_{k - 1} + \dots + \cancelto{1}{10} \cdot n_{1} + n_{0}. \\
        &\stackrel{(*)}{\equiv} n_{k} + n_{k - 1} + \dots + n_{1} + n_{0} \pmod{3}.
    \end{aligned}$$

\subsection*{Tricks}
    Let's find $10^{2022} \pmod{7}$.
    
    First note that $\{10^{k}\}_{k \ge 1} \pmod{7}$ is periodic, as we will show below.
    \begin{align*}
        10^{1} &\equiv 3 \pmod{7} \\
        10^{2} &\equiv 2 \pmod{7} \\
        10^{3} &\equiv 10^{2} \cdot 10 \equiv 2 \cdot 3 = 6 \pmod{7} \\
        10^{4} &\equiv 10^{3} \cdot 10 \equiv 6 \cdot 3 \equiv 4 \pmod{7} \\
        10^{5} &\equiv 10^{4} \cdot 10 \equiv 4 \cdot 3 \equiv 5 \pmod{7} \\
        10^{6} &\equiv 10^{5} \cdot 10 \equiv 5 \cdot 3 \equiv 1 \pmod{7} \\
        10^{7} &\equiv 10^{6} \cdot 10 \equiv 1 \cdot 3 = 3 \pmod{7}
    \end{align*}
    The sequence $3, 2, 6, 4, 5, 1$ will repeat as we take larger powers of $10$.
    
    \vspace{1.5mm} 
    Now $10^{2022} = 10^{6 \cdot q} \equiv 1 \pmod{7}$, since the sequence repeats every $6$ values.
    
\subsection*{Fermat's Little Theorem}
    To introduce this theorem, we first state a lemma.
    
    \vspace{1.5mm}
    \textbf{Lemma} Take some prime number $p$ and some integer $a$. Then $\{i \cdot a\}_{1 \le i \le p - 1}$ are all diff $\pmod{p}$
    
    \textit{Proof} (By contradiction) Assume there are $1 \le i, j \le p - 1$ such that $ia \equiv ja \pmod{p}$ OR $(i - j)a \equiv 0 \pmod{p}$ contradiction! (unless $i = j$).
    
    Using this lemma, Fermat observed that $(*)$
    \begin{align*}
        a^{p - 1}(p - 1)! &\equiv (p - 1)! \pmod{p} \\
        (a^{p - 1} - 1)(p - 1)! &\equiv 0 \pmod{p}
    \end{align*}
    Finally, he created his own theorem, stated below.
    
    \vspace{1.5mm}
    \textbf{Fermat's Little Theorem.} For any prime number $p$ and integer $a$, $a^{p - 1} \equiv 1 \pmod{p}$.
    
    % There is also a sub-theorem ??? you do not need to know, whose proof is stated below.
    
    % \vspace{1.5mm} \textit{Proof. }
    % B\'ezout's Identity states that if $d = \gcd(a, b)$, then there exists integers $r, s$ such that $d = ra + sb$.
    
    % Now suppose $d = 1$. Then 
    % \begin{align*}
    %     1 &= \gcd(a, b) \\
    %     1 &= r \cdot a + s \cdot b \\
    %     c &= (r \cdot c)a + s(bc) \\
    %     c &= (r \cdot c)a + s \cdot(q) = a(r \cdot c + s \cdot q)
    % \end{align*}

\subsection*{Chinese Remainder Theorem}
    Let $m_{1}, m_{2}, \dots, m_{k} \in \mathbb{N}^{*}$ such that every distinct pair $m_{i}, m_{j}$ is pairwise coprime. Also consider any $a_{1}, a_{2}, \dots, a_{k} \in \mathbb{Z}$.
    
    There exists exactly one $x$ (taken mod $m$) such that 
    \begin{align*}
        x &\equiv a_{1} \pmod{m_{1}} \\
        x &\equiv a_{2} \pmod{m_{2}} \\
        &\vdotswithin{ = } \\
        x &\equiv a_{k} \pmod{m_{k}}
    \end{align*}
    
    \textit{Proof. } We construct a solution $$M_{i} = \frac{M}{m_{i}}.$$ First set $y_{i} \in \mathbb{Z}$ such that $y_{i} \cdot M_{i} \equiv 1 \pmod{m_{i}}$.
    Then set $$x = a_{1}y_{1}M_{1} + a_{2}y_{2}M_{2} + \dots + a_{i - 1}y_{i - 1}M_{i - 1} + a_{i}y_{i}M_{i} + a_{i + 1}y_{i + 1}M_{i + 1} + \cdots + a_{k}y_{k}M_{k}.$$
    Claim: $x \equiv a_{i} \pmod{m_{i}}$, because every term goes to zero except $a_{i}y_{i}M_{i}$, which goes to $a_{i}$.
    
\end{document}