\documentclass{article}
\usepackage[margin=1in]{geometry}
\usepackage{amsmath, amssymb, amsthm}
\usepackage{enumitem}

\usepackage{tcolorbox}

%Formatting and Spacing
\setitemize[1]{noitemsep, parsep = 5pt, topsep = 5pt}
\setenumerate[1]{label = (\alph*), parsep = 1pt, topsep = 5pt}
\setlength\parindent{0pt}
\linespread{1.1}

%Custom Title Fields
\newcommand{\lectTitle}{Lecture 8 Notes}
\newcommand{\lectTime}{February 7, 2022}
\newcommand{\lectClass}{Honors Discrete Mathematics}
\newcommand{\lectClassInstructor}{Gerandy Brito}
\newcommand{\lectSection}{Spring 2022}
\newcommand{\lectAuthorName}{Sarthak Mohanty}

%Headers and Footers
\usepackage{fancyhdr}
\usepackage{extramarks}
\pagestyle{fancy}
\lhead{\lectTime}
\chead{\lectClass \ (\lectClassInstructor)}
\rhead{\lectTitle}
\cfoot{\thepage}
\renewcommand\headrulewidth{0.4pt}
\renewcommand\footrulewidth{0.4pt}

\title{
    \vspace{2in}
    \textbf{\lectClass:\\ \lectTitle}\\
    \vspace{0.1in}\large{\textit{\lectClassInstructor\ \lectSection}}
    \vspace{3in}
    \author{\textbf{\lectAuthorName}}
    \date{}
}

\begin{document}

\maketitle
\pagebreak

\section*{Induction}
    Induction is a proof strategy where we aim to prove a statement of the form $(\forall n \in \mathbb{N})(P(n))$. It works as follows:
    \begin{itemize}
        \item Prove $P(1)$ is true.
        \item Prove that for all $n$, if $P(n)$ is true, then $P(n + 1$ is true; in other words, prove $(\forall n)(P(n) \Rightarrow P(n + 1))$.
    \end{itemize}
    
    $\star$ Explain further:

\subsection*{Example 1}
    We begin with what is largely considered the most classical inductive proof.
    
    Prove using induction that for any $n \in \mathbb{N}$, $$1 + 2 + 3 + \dots + n = \frac{n(n + 1)}{2}.$$

\subsection*{Solution}
    Let $P(n)$ be the sentence $$1 + 2 + 3 + \dots + n = \frac{n(n + 1)}{2}.$$
    \textsc{Base Case}: $P(1)$ is true, since the LHS of the equation is $1$ and the RHS of the equation is $1 = \frac{1(1 + 1)}{2} = 1.$
    \textsc{Inductive Step}: Now let $n \in \mathbb{N}$ such that $P(n)$ is true. We wish to prove $P(n + 1)$ is true as well; in other words, we wish to show $$1 + 2 + \dots + n + (n + 1) = \frac{(n + 1)(n + 2)}{2}.$$ We focus on the LHS. Note that by the inductive hypothesis, $$(1 + 2 + \dots + n) + (n + 1) = \frac{n(n + 1)}{2} + (n + 1) = \frac{n(n + 1)}{2} + \frac{2(n + 10)}{2} = \frac{(n + 2)(n + 1)}{2}.$$
    Hence $P(n + 1)$ is true as well. \\
    \textsc{Conclusion}: Therefore by induction, for all $n \in \mathbb{N}$, $P(n)$ is true.
    
    \vspace{1.5mm}
    \textbf{TA Remark.} You'll encounter this closed form many, many times throughout your undergraduate career. Make sure to memorize it.

\subsection*{Example 2}
    Prove using induction that for any $n \in \mathbb{W}$, $$1 + r + r^{2} + r^{3} + \dots + r^{n}= \frac{1 - r^{n + 1}}{1 - r}.$$

\subsection*{Solution}
    Let $P(n)$ be the sentence $$1 + r + r^{2} + r^{3} + \dots + r^{n}= \frac{1 - r^{n + 1}}{1 - r}.$$
    \textsc{Base Case}: $P(0)$ is true, since the LHS of the equation is $r^{0} = 1$ and the RHS of the equation is $\frac{1 - r^{1}}{1 - r} = 1.$
    \textsc{Inductive Step}: Now let $n \in \mathbb{W}$ such that $P(n)$ is true. We wish to prove $P(n + 1)$ is true as well; in other words, we wish to show $$1 + r + r^{2} + r^{3} + \dots + r^{n} + r^{n + 1} = \frac{1 - r^{n + 2}}{1 - r}.$$ We focus on the LHS. Note that by the inductive hypothesis, $$(1 + r + r^{2} + \dots + r^{n}) + r^{n + 1} = \frac{1 - r^{n + 1}}{1 - r} + r^{n + 1} = \frac{1 - r^{n + 1}}{1 - r} + \frac{r^{n + 1} - r^{n + 2}}{1 - r} = \frac{1 - r^{n + 2}}{1 - r}.$$
    Hence $P(n + 1)$ is true as well. \\
    \textsc{Conclusion}: Therefore by induction, for all $n \in \mathbb{N}$, $P(n + 1)$ is true.
    
    \vspace{1.5mm}
    \textbf{TA Remark.} This closed form is important to memorize as well; however, as the formula is rather complex, it may be difficult to memorize. One trick I use to memorize complex formulas is to memorize the \underline{derivation} of the respective formula, shown below:
    
    %%%%%%%%%%%%%%%%%%%%%%%%%%%%%%%%%%%%%%%%%%%%%%%%%%%%%%%%%%%%%%%%%%

\subsection*{Example 3}
    Prove using induction that for any $n \ge 4$, $$2^{n} \le n!.$$

\subsection*{Solution}
    Let $P(n)$ be the sentence $$2^{n} \le n!.$$
    \textsc{Base Case}: $P(4)$ is true, since the LHS of the equation is $2^4 = 16$ and the RHS of the equation is $4!=24.$
    \textsc{Inductive Step}: Now let $n \in \mathbb{W}$ such that $P(n)$ is true. We wish to prove $P(n + 1)$ is true as well; in other words, we wish to show $$2^{n+1}\leq (n+1)!$$. We focus on the LHS. Note that by the inductive hypothesis, $$2^{n+1} =2\cdot 2^n \leq 2\cdot n! < (n+1)\cdot n!=(n+1)!$$
    where we used that $2<4\leq n$. Hence $P(n + 1)$ is true as well. \\
    \textsc{Conclusion}: Therefore by induction, for all $n \in \mathbb{N}$, $P(n)$ is true.


\subsection*{Example 4 (TBC)}
    Let $n \in \mathbb{W}$. Define the harmonic numbers $H_{n}$, as $$H_{n} = \frac{1}{1} + \frac{1}{2} + \frac{1}{3} + \frac{1}{n}.$$ Use induction to prove $$H_{2^{n}} \ge 1 + \frac{n}{2}.$$

\subsection*{Solution}
    Let $P(n)$ be the sentence $$H_{2^{n}} \ge 1 + \frac{n}{2}.$$
    \textsc{Base Case}: $P(0)$ is true, since $H_{2^{0}} = H_{1} = 1 \ge 1 + \frac{0}{2} = 1$. \\
    \textsc{Inductive Step}: Now let $n \in \mathbb{W}$ such that $P(n)$ is true. We wish to prove $P(n + 1)$ is true as well; in other words, we wish to show $$H_{2^{n + 1}} \ge 1 + \frac{n + 1}{2}.$$ We focus on the LHS. Note that
    $$\begin{aligned}[t]
        H_{2^{n + 1}} &= 1 + \frac{1}{2} + \dots + \frac{1}{2^{n}} + \frac{1}{2^{n} + 1} + \dots + \frac{1}{2^{n + 1}} \\
        &= H_{2^{n}} + \frac{1}{2^{n} + 1} + \dots + \frac{1}{2^{n + 1}} \\
        &\ge \left(1 + \frac{n}{2}\right) + \frac{1}{2^{n} + 1} + \dots + \frac{1}{2^{n + 1}} \\
        &\ge \left(1 + \frac{n}{2}\right) + 2^{n} \cdot \frac{1}{2^{n + 1}} \text{(Explain why)} \\ 
        &= \left(1 + \frac{n}{2}\right) + \frac{1}{2} = 1 + \frac{n + 1}{2}.
    \end{aligned}$$
    Hence $P(n + 1)$ is true as well. \\
    \textsc{Conclusion}: Therefore by induction, for all $n \in \mathbb{N}$, $P(n + 1)$ is true.



\subsection*{Example 5}
    Let $n$ be a positive integer. Show that every $2^{n} \times 2^{n}$ checkerboard with one square removed can be tiled by triominos (shown below).
    
\subsection*{Solution (TBC)}
     Let $P(n)$ be the sentence $$\text{Every $2^{n} \times 2^{n}$ checkerboard with one square removed can be tiled by triominos}.$$
    \textsc{Base Case}: $P(1)$ is true, since every $2 \times 2$ checkerboard with one square removed can be tiled with a single triomino (shown below). \\
    \textsc{Inductive Step}: Now let $n \in \mathbb{N}$ such that $P(n)$ is true. Now consider a $2^{n + 1} \times 2^{n + 1}$ checkerboard. We can split it into four quadrants. For each of these quadrants, by the inductive hypothesis, we can tile each of them, leaving only one square removed. WLOG, suppose the one square removed is as shown in the picture below. Then, the three squares removed can be filled with a triomino, leaving the entire checkerboard filled.

\subsection*{Example 6}
    There is a group of $n \ge 2$ people standing on a field. The distances between each pair are distinct numbers. Every person has a ball. At the same time, they each throw their respective balls to the closest person next to them. Show that at least one person is left with no ball.

\subsection*{Solution}


\section*{Post Lecture}

\subsection*{Question 1 (Calculus)}
    Prove by induction that for each $n \in \mathbb{W}$, $$\int_0^\infty x^ne^{-x}dx = n!.$$ You may use the fact that for each whole number $k$, $\lim_{x \to \infty} x^ke^{-x} = 0$. (proven by applying L'Hopital's rule $k$ times.)

\subsection*{Solution}
    Let $P(n)$ be the sentence $$\int_0^\infty x^ne^{-x}dx = n!.$$
    \textsc{Base Case}: $P(0)$ is true, since $$\int_0^\infty x^0e^{-x}dx = \lim_{a \to \infty} \int_0^a e^{-x}dx = \lim_{a \to \infty}\left[ -e^{-x} \right]_0^a = e^0 - 0 = 1 = 0!.$$
    \textsc{Inductive Step}: Now let $n \in \mathbb{W}$ such that $P(n)$ is true. Then using integration by parts, we have $$\int_0^\infty x^{n + 1}e^{-x}dx = \lim_{a \to \infty} \left[ -x^{n + 1}e^{-x}\right]_0^\infty + (n + 1)\int_0^\infty x^ne^{-x}dx = 0 + (n + 1)(n!) = (n + 1)!$$ Hence $P(n + 1)$ is true as well. \\
    \textsc{Conclusion}: Therefore by induction, for all $n \in \mathbb{W}$, $P(n)$ is true.



\end{document}