\documentclass{article}
\usepackage[margin=1in]{geometry}
\usepackage{amsmath, amssymb, amsthm}
\usepackage{enumitem}
\usepackage{xcolor}

%Formatting and Spacing
\setitemize[1]{noitemsep, parsep = 5pt, topsep = 5pt}
\setenumerate[1]{label = (\alph*), parsep = 1pt, topsep = 5pt}
\setlength\parindent{0pt}
\linespread{1.1}

%Custom Title Fields
\newcommand{\lectTitle}{Lecture  Notes}
\newcommand{\lectTime}{January 31, 2022}
\newcommand{\lectClass}{Honors Discrete Mathematics}
\newcommand{\lectClassInstructor}{Tillson Galloway}
\newcommand{\lectSection}{Spring 2022}
\newcommand{\lectAuthorName}{Gerandy Brito \& Sarthak Mohanty}

%Headers and Footers
\usepackage{fancyhdr}
\usepackage{extramarks}
\pagestyle{fancy}
\lhead{\lectTime}
\chead{\lectClass \ (\lectClassInstructor)}
\rhead{\lectTitle}
\cfoot{\thepage}
\renewcommand\headrulewidth{0.4pt}
\renewcommand\footrulewidth{0.4pt}

\title{
    \vspace{2in}
    \textbf{\lectClass:\\ \lectTitle}\\
    \vspace{0.1in}\large{\textit{\lectClassInstructor\ \lectSection}}
    \vspace{3in}
    \author{\textbf{\lectAuthorName}}
    \date{}
}

\begin{document}

\maketitle
\pagebreak

Recall definition of relation:\\

{\bf Def:} a relation is a subset of the cartesian product $A\times B$.

We denote relations by $\mathcal{R}$. We write $a\mathcal{R} b$ as an infix to indicate that $(a,b)\in \mathcal{R}$ (i.e.: in the subset denoted by $\mathcal{R}$). When $A=B$ we said that $\mathcal{R}$ is a relation on $A$.\\

Examples: \textcolor{blue}{you can illustrate these examples more visually, so they see it as subsets of the cartesian product $\mathbb{N}^2$.}\\

\begin{itemize}
    \item $\mathcal{R}_1=\{(a,b~|~ a\leq b)\}$
    \item $\mathcal{R}_2=\{(a,b~|~ a=b)\}$
    \item $\mathcal{R}_3=\{(a,b~|~ a+b\leq 2022)\}$
    \item $\mathcal{R}_4=\{(a,b~|~ a \mbox{ divides } b)\}$
\end{itemize}

{\bf Def:} A function is a relation on $A\times B$ such that 
\[
\forall a\in A~\exists!~ b\in B: a\mathcal{R}b.
\]

\textcolor{blue}{Clarify the symbol ``there exists exactly one'' as it may be new to many. }

Often we are presented with functions describing the unique element $b$ for each $a$. In this case we use $f$ to denote the function and write $b=f(a)$.\\

Notation:\\

$A$ is called the domain. $B$ is called codomain. The range of $f$ is the set of all $b\in B$ for which there is at least one $a\in A$ satisfying $f(a)=b$. \textcolor{blue}{Now make then think about the representation using relations!} 

{\bf Def:} Injetive (or one-to-one), Surjective (or onto). Bijections (\textcolor{blue}{Important to introduce cardinality later!}).\\

Examples: \textcolor{blue}{for each example, make sure they understand what the domain and codomain is. Informally discuss injectivity and surjectivity.}

\begin{itemize}
    \item $f$ assigns to each student in the class its height in cm.
    \item f(x)=x+1 from $\mathbb{N}$ to itself, then from $\mathbb{Z}$ to itself.
    \item $f$ assigns to each bit string of length two or more its last two bits.
    \item $f$ assigns to each real number the largest integer less or equal than the number (\textcolor{blue}{this is the floor, of course. Introduce the standard notation}) 
    
\end{itemize}

\textcolor{blue}{Now discuss what it means to be one-to-one, onto, in the language of relations. Write the definitions. Mention that, in practice, we get functions not as relations and work directly with the $f$ description. Now give then the general strategy to prove a function is one-to-one:}
\[
\mbox{$f$ is one-to-one if } f(x)=f(y)\rightarrow x=y.
\]

\textcolor{blue}{and onto:}
\[
\mbox{$f$ is onto if } \forall b\in B, ~\exists a\in A:~f(a)=b. 
\]

\textcolor{blue}{you can now go back to the examples and check these properties more formally. Do it only after judging on the time and if you think students will appreciate it.}

So why relations? They are more general and allow us to study more complex sets.\\

Properties:

\begin{itemize}
    \item \underline{Reflexive:} $\forall a\in A~(a\mathcal{R}a)$
    \item \underline{Symmetric:} $\forall a, b\in A~(a\mathcal{R}b \mbox{ iff } b\mathcal{R}a)$
    \item \underline{Antisymmetric:} $\forall a, b\in A~(a\mathcal{R}b \wedge b\mathcal{R}a \rightarrow a=b)$
    \item \underline{Transitive:} $\forall a, b, c\in A~(a\mathcal{R}b \wedge b\mathcal{R}c \rightarrow a\mathcal{R}c)$
\end{itemize}

\textcolor{blue}{Let then realize by themselves that these only work if $A=B$! Help then visualize what it means to be reflexive (the main diagonal is in $\mathcal{R}$); symmetric ($\mathcal{R}$ is symmetric with respect to the main diagonal); and antisymmetric  (there are no symmetric points!)}

For each of the examples at the beginning, check if they satisfy these properties.\\

{\bf Def:} A relation is said to be an equivalence relation if it is reflexive, symmetric, an transitive.\\

\textcolor{blue}{From the examples, which ones are equivalence relations?}\\

Example: Show that the relation on the real numbers defined as $a\mathcal{R}b \leftrightarrow a-b\in \mathbb{Z}$ is an equivalence relation.\\

{\bf Def:} Given a set $A$ and an equivalence relation on it, the equivalence classes are the sets
\[
[a]_{\mathcal{R}}=\{x\in A~|~a\mathcal{R} x\}
\]
\textcolor{blue}{Go back to the examples of equivalence relation and informally describe the equivalence classes.}

{\bf Theorem.} Let $A$ be a set and $\mathcal{R}$ an equivalence relation on it. The following are equivalent:
\begin{itemize}
    \item[(i)] $a\mathcal{R}b$.
    \item[(ii)] $[a]_{\mathcal{R}}=[b]_{\mathcal{R}}$.
    \item[(iii)] $[a]_{\mathcal{R}}\cap [b]_{\mathcal{R}}\neq \emptyset$.
\end{itemize}

\textcolor{blue}{Take your time to prove this. Point out the subtle conclusion that if two classes do have an intersection, they must be equal.Lead them to conclude that the equivalence classes form a partition of the set $A$. Some questions worth asking: why is every element in a class? Ans: because of the reflexive property. Why it is a partition? Because of facts (ii) and (iii) of the Theorem.}

{\bf Def:} The cardinality of a set $S$ is
\begin{itemize}
    \item the number of elements in $S$, if it is finite.
    \item $\infty$ otherwise.
\end{itemize}

\textcolor{blue}{Now it is a good time to test how are they with this with questions like Do sets with the same cardinality have the same number of elements? What about the natural numbers vs. the integers? Or vs. the real numbers? Feel free to mention the intuitive and misleading idea that ``there are more integers than naturals since the first includes the negative numbers'' etc.}

{\bf Def:} Two sets $A$ and $B$ are said to have the same cardinality if there exists a bijection from $A$ to $B$.

Now consider the collection of all sets (i.e.: the power set of the universe!) and define the relation
\[
A\mathcal{R} B~\mbox{ if $A$ and $B$ have the same cardinality.}
\]

Check that $\mathcal{R}$ is an equivalence relation. \textcolor{blue}{Another proof to do! If time allows, you can dive into the countable and uncountable idea. Start by asking the simple question of whether $\mathbb{N}$ and $\mathbb{Z}$ have the same cardinality. ASk the students for ideas and show a bijection. Other examples to consider: the set of even numbers, the set of odd numbers. If you get to this stage, officially define what it means to be countable:}

{\bf Def:} A set is said to be countable if it is finite or has the same cardinality as the natural numbers.


\section*{Post Lecture}

Al and Bob play a game. They have the numbers 1,2,...,9 written on cards face up. Players alternate taking cards. The first player to have exactly 3 cards whose sum is 15 wins. Determine which player, if any has a winning strategy.

Solution: One can show that the 8 possible ways to win this game biject to the 8 possible ways to win Tic-Tac-Toe on a magic square, and Tic-Tac-Toe is known to have no winning strategy for either player.


4 9 2
3 5 7
8 1 6
\end{document}