\documentclass[11pt,fleqn]{article}

\usepackage[latin1]{inputenc}
\usepackage{enumerate}
\usepackage[hang,flushmargin]{footmisc}
\usepackage{amsmath}
\usepackage{amsfonts}
\usepackage{amssymb}
\usepackage{amsthm}

\theoremstyle{definition}

\newtheorem{theorem}{Theorem}[section]
\newtheorem{lemma}[theorem]{Lemma}
\newtheorem{corollary}[theorem]{Corollary}
\newtheorem{proposition}[theorem]{Proposition}
\newtheorem{definition}[theorem]{Definition}
\newtheorem{example}[theorem]{Example}

\setlength{\oddsidemargin}{0px}
\setlength{\textwidth}{460px}
\setlength{\voffset}{-1.5cm}
\setlength{\textheight}{20cm}
\setlength{\parindent}{0px}
\setlength{\parskip}{10pt}

\title{CS 2051: Project Title}
\date{2017\\ December}
\author{Author 1 \\ Georgia Institute of Technology
\and Author 2 \\ Georgia Institute of Technology}

%%%%%%%%%%%%%%%%%%%%%%%%%%%%%%%%%%%%%%%%%%%%%%%%%%%%%%%%%%%%%%%%

\begin{document}

\maketitle

\begin{abstract} \noindent
% Here you write a short summary of what the document contains, 
% like the blurb on the back of a book.

\end{abstract}

\tableofcontents

\newpage

\section{Background}
% This section should develop the basic definitions and
% preliminary results required for the statement and proof of
% your main result, including examples that help the reader to
% develop intuition for the concepts presented.



\section{Main result}
% In this section, you should state and prove your main result,
% and provide some basic consequences and examples that help the
% reader to understand it. You may want to change this section's
% name to something more informative.



\section{Extension/application/generalisation}
% You should change this section's name to something relevant to
% your project (e.g. "Applications of the RSA encryption
% algorithm" or "Linear Diophantine equations in $n$ unknowns"



\section{Reflection/conclusion/future directions}
% You should change this section's name to something relevant to
% your project. This section allows you to reflect or conclude
% the work you have done in your project and discuss future work
% that could be done on the project (e.g. applications you tried
% to make but couldn't find the time).



\section{Acknowledgements and references}
% Here you should acknowledge people whose help you are thankful
% for (and why), and any sources such as books and websites that
% you used when studying for the project.




\end{document}