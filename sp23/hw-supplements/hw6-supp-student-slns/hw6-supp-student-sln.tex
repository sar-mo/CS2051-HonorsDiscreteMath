\documentclass{article}
\usepackage[margin=1in]{geometry}
\usepackage{amsmath, amssymb, amsthm}
\usepackage{enumitem}

\newenvironment{solution}
{
\par
\medskip
\color{blue}
\textbf{Solution:}
}
{
\medskip
\par
}
%Highlighting
\usepackage{xcolor, soul}
\sethlcolor{lightgray}

%Cases Environment
\newlist{Cases}{enumerate}{3}
\setlist[Cases]{leftmargin = .25in, label = {Case \arabic*.}, topsep = 0.01in, itemsep = 0.04in, itemindent = .5in, parsep = 0in}

%Formatting and Spacing
\setitemize[1]{noitemsep, parsep = 5pt, topsep = 5pt}
%\setenumerate[1]{label = (\alph*), parsep = 1pt, topsep = 5pt}
\setlength\parindent{0pt}
\linespread{1.15}

% title
\title{\vspace{-1cm}CS 2051: Honors Discrete Mathematics \\Spring 2023 Homework 6 Supplement}
\author{Sean Peng\footnote{Solutions were published with the permission of the student.}}
\date{}

\begin{document}

\maketitle
\begin{enumerate}
    \item A countable set is a set that has a one-to-one mapping with the set of natural numbers. Prove that the set of positive rational numbers is countable by setting up a function that assigns to a rational number $p/q$ with $\gcd (p, q) = 1$ the base 11 number formed by the decimal representation of p followed by the base 11 digit A, which corresponds to the decimal number 10, followed by the decimal representation of q. 
    \begin{solution}
        By the definition of rational numbers, for all $x \in \mathbb{Q}$, there exists $p, q \in \mathbb{Z}$, such that $x = p / q$ and $ \gcd(p, q) = 1$. Let $f: \mathbb{Q}^+ \rightarrow \{0, 1, 2, 3, 4, 5, 6, 7, 8, 9, A\}^n$, such that $f(p / q) = (pAq)_{11}$, where $\gcd(p, q) = 1$ and $\{0, 1, 2, 3, 4, 5, 6, 7, 8, 9, A\}^n$ is the set of base 11 numbers. We will show that this function is one-to-one.

        \medskip
        Let $x, y \in \mathbb{Q}^+$, where $f(x) = f(y)$. By the definition of rational numbers, we have $x = p / q, y = r / s$, where $p, q, r, s \in \mathbb{Z}^+$ and $\gcd(p, q) = \gcd(r, s) = 1$. By the definition of $f$, $pAq = rAs$. Since $p, q, r, s$ do not contain the digit $A$, we have $p = r, q = s$. Then, $p / q = r / s$ and $x = y$. We have shown that if $f(x) = f(y)$, then $x = y$. Therefore, $f$ is one-to-one.

        \medskip
        If $p, q > 0$, then $(pAq)_{11} > 0$. Since every element in the range of $f$ is positive, and each positive base 11 number converts to a different positive decimal number, there is also a one-to-one mapping from $\{0, 1, 2, 3, 4, 5, 6, 7, 8, 9, A\}^n$ to $\mathbb{N}$. Then, there is a one-to-one mapping of $|\mathbb{Q}^+|$ to $\mathbb{N}$. Therefore, we have proven that $\mathbb{Q}^+$ is countable. $\blacksquare$
    \end{solution}

    \item Define a Carmichael number as a composite number n which satisfies the following relation: $b^n \equiv b \pmod{n},$ for all integers b.
    Show that if $n = p_1p_2 \cdots p_k$, where $p_1, p_2, \dots, p_k$ are distinct primes that satisfy $p_j - 1 \vert n - 1$ for $j = 1, 2, \dots, k,$ then $n$ is a Carmichael number.
    \begin{solution}
        I proceed with a direct proof. We will show that if $n = p_1p_2 \cdots p_k$, where $p_1, p_2, \dots, p_k$ are distinct primes that satisfy $p_j - 1 \vert n - 1$ for $j = 1, 2, \dots, k,$ then $n$ is a Carmichael number.

        \medskip
        Let $b \in \mathbb{Z}$. WLOG, let $p_j$ be an arbitrary prime factor of $n$, where $1 \leq j \leq n$. We will consider two cases: $\gcd(b, p_j) = 1$ and $\gcd(b, p_j) > 1$.

        \medskip
        Suppose $\gcd(b, p_j) = 1$. By Fermat's Little Theorem, $b^{p_j - 1} \equiv 1 \pmod{p_j}$. Since $p_j - 1 \vert n - 1$, there exists a constant $c$ such that $n - 1 = c(p_j - 1)$. Then, 
        \begin{alignat*}{4}
             &b^{(p_j - 1)c} \equiv 1^c &&\pmod{p_j} &&\quad &&\text{Raise both sides to the power of $c$}\\
             &b^{n - 1} \equiv 1 &&\pmod{p_j} &&\quad &&\text{Simplify both sides}\\
             &b^{n} \equiv n &&\pmod{p_j} &&\quad &&\text{Multiply both sides by n}
         \end{alignat*}
         Suppose $\gcd(b, p_j) > 1$. Then, since $p_j$ is prime, we have $p_j \vert b$, which leads to: 
         \begin{alignat*}{4}
             &b \equiv 0 &&\pmod{p_j} &&\quad &&\text{Definition of mod}\\
             &b^{n} \equiv 0 &&\pmod{p_j} &&\quad &&\text{Raise both sides to the power of $n$}\\
             &b^{n} \equiv b &&\pmod{p_j} &&\quad &&\text{From line 1 and 2}
         \end{alignat*}
         We have shown that for all $j, 1 \leq j \leq n$, $b^{n} \equiv b \pmod{p_j}$, for all $b \in \mathbb{Z}$. Since $n = p_1p_2 \cdots p_k$ where $p_1, p_2, \dots, p_k$ are distinct, by the Chinese Remainder Theorem, $b^{n} \equiv b \pmod{n}$, for all $b \in \mathbb{Z}$. Therefore, with the given conditions, $n$ is a Carmichael number. $\blacksquare$
    \end{solution}
\end{enumerate}

\end{document}