\documentclass{article}
\usepackage[margin=1in]{geometry}
\usepackage{amsmath, amssymb, amsthm}
\usepackage{enumitem}

% colored links
\usepackage{hyperref}
\hypersetup{
    colorlinks=true,
    linkcolor=blue,
    filecolor=magenta,      
    urlcolor=blue,
    }



% Inputting Python code
\usepackage[dvipsnames]{xcolor}
\definecolor{textblue}{rgb}{.2,.2,.7}
\definecolor{textred}{rgb}{0.54,0,0}
\definecolor{textgreen}{rgb}{0,0.43,0}
\usepackage{upquote}
\usepackage{listings}
\lstset{
    language=Python, 
    tabsize=4,
    basicstyle={\ttfamily},
    keywordstyle=\color{textblue},
    commentstyle=\color{textgreen},
    stringstyle=\color{textred},
    frame=none,
    columns=fullflexible,
    keepspaces=true,
    showstringspaces=false,
    xleftmargin=-15mm, % manual adjustment, figure out permanent solution
}
\usepackage{tcolorbox}
\tcbuselibrary{skins,hooks}
\usetikzlibrary{shadows}
\usepackage{lipsum}

%Images
\usepackage{graphicx}
    \usepackage{subcaption}
    \usepackage{float}

%Formatting and Spacing
\setitemize[1]{noitemsep, parsep = 5pt, topsep = 5pt}
\setenumerate[1]{label = (\alph*), parsep = 1pt, topsep = 5pt}
\setlength\parindent{0pt}
\linespread{1.1}

% title
\title{\vspace{-1cm}CS 2051: Honors Discrete Mathematics \\Spring 2023 Homework 7 Supplement}
\author{Sarthak Mohanty }
\date{}

\begin{document}

\maketitle


Title: ECC is the new RSA

See now I think this is going to be \textbf{really interesting} because it's not covered in the textbook at all, and in fact ECC is not really ever taught in a discrete math course because historically we just used RSA / instructors never took the time to introduce it in a number theory context.

They'll have to download Jupyter Notebook for this one, give instructions or make them use google collab


Spitting ideas:

\begin{itemize}
    \item introduce disclaimer that I'm not security expert.
    \item Talk about RSA (prolly learnt in class that week)
    \item Talk about vulnerabilities quadratic sieve , look at cloudflare
    
    % These factoring algorithms get more efficient as the size of the numbers being factored get larger. The gap between the difficulty of factoring large numbers and multiplying large numbers is shrinking as the number (i.e. the key's bit length) gets larger. As the resources available to decrypt numbers increase, the size of the keys need to grow even faster. This is not a sustainable situation for mobile and low-powered devices that have limited computational power. The gap between factoring and multiplying is not sustainable in the long term.
    \item Talk about how we need to redefine how we think about about this stuff
    \item By redefine means we need to back up and look at it from a broader point of view
    \item Talka bout symmetric key encryption in general
    \item Talk about trapdoor functions


    \item Now finally we introduce ECC. This is part 1
    \item We introduce elliptic curve graph (weierstrass something something)
    \item Make them do the scalar addtion and multiplication.

    \item Now move on to part 2. Introduce congruency classes./
    \item make them do the same thing over gaussian field.

    \item idea, but hard. can make them actually do ecc and fill in the gaps using this. Problem is on my end, actually implementing ecc finna take hella time.
    \item another idea; introduce it hrough diffie helman key exchange (very intuitive, see youtube video in outline) them students can see one to one relation. maybe cover evne less RSA then I originally thought.

    \item aside. make them see in practice (use google to check security of website, any good website should be using ECC like proton or cloudlfare)
    \item aside 2. if really uninspired just make them do something with RSA (maybe something small but imma do mostly new stuff)
\end{itemize}

    

\end{document}