\documentclass{article}
\usepackage[margin=1in]{geometry}
\usepackage{amsmath, amssymb, amsthm}
\usepackage{enumitem}

% colored links
\usepackage{hyperref}
\hypersetup{
    colorlinks=true,
    linkcolor=blue,
    filecolor=magenta,      
    urlcolor=blue,
    }



% Inputting Python code
\usepackage[dvipsnames]{xcolor}
\definecolor{textblue}{rgb}{.2,.2,.7}
\definecolor{textred}{rgb}{0.54,0,0}
\definecolor{textgreen}{rgb}{0,0.43,0}
\usepackage{upquote}
\usepackage{listings}
\lstset{
    language=Python, 
    tabsize=4,
    basicstyle={\ttfamily},
    keywordstyle=\color{textblue},
    commentstyle=\color{textgreen},
    stringstyle=\color{textred},
    frame=none,
    columns=fullflexible,
    keepspaces=true,
    showstringspaces=false,
    xleftmargin=-15mm, % manual adjustment, figure out permanent solution
}
\usepackage{tcolorbox}
\tcbuselibrary{skins,hooks}
\usetikzlibrary{shadows}
\usepackage{lipsum}

%Images
\usepackage{graphicx}
    \usepackage{subcaption}
    \usepackage{float}

%Formatting and Spacing
\setitemize[1]{noitemsep, parsep = 5pt, topsep = 5pt}
\setenumerate[1]{label = (\alph*), parsep = 1pt, topsep = 5pt}
\setlength\parindent{0pt}
\linespread{1.1}

% title
\title{\vspace{-1cm}CS 2051: Honors Discrete Mathematics \\Spring 2023 Homework 7 Supplement}
\author{Sarthak Mohanty }
\date{}

\begin{document}

\maketitle


Title: ECC is the new RSA

\textbf{Note: only one part should be about full ECC implementation, the other parts should be about number theory in general, since the whole idea is covering mathematical concepts anyway.}

\section*{Overview}

Elliptic Curve Cryptography (ECC) is one of the most powerful cryptosystems in use today. Companies are using ECC everything to secure everything from our customers' HTTPS connections to how we pass data between our data centers. In fact, based on currently understood mathematics, ECC provides a significantly more secure foundation than first generation public key cryptography systems like RSA, for reasons we'll explore in detail later.

\vspace{2mm}
See now I think this supplement is going to be \textbf{really interesting} because it's not covered in the textbook at all, and in fact ECC is not really ever taught in a discrete math course because most people think it's too complicated to teach.

\vspace{2mm}
For this supplement, you'll be working with a Jupyter Notebook (.ipynb) file. To open it, you can either install Jupyter Notebook locally, or you can use Google Colab to work in the browser.




\subsection*{Part 1: Assymetric Key Exchanges}


\subsection*{Generalizing RSA: Trapdoor Functions}

\begin{tcolorbox}
% [colback=yellow!30]
    In this part, you'll fully implement the Diffie-Helman key exchange.
    \end{tcolorbox}


\vspace{2mm}






Spitting ideas:

\begin{itemize}
    \item introduce disclaimer that I'm not security expert.
    \item Talk about RSA (prolly learnt in class that week)
    \item Talk about vulnerabilities quadratic sieve , look at cloudflare
    
    % These factoring algorithms get more efficient as the size of the numbers being factored get larger. The gap between the difficulty of factoring large numbers and multiplying large numbers is shrinking as the number (i.e. the key's bit length) gets larger. As the resources available to decrypt numbers increase, the size of the keys need to grow even faster. This is not a sustainable situation for mobile and low-powered devices that have limited computational power. The gap between factoring and multiplying is not sustainable in the long term.
    \item Talk about how we need to redefine how we think about about this stuff
    \item By redefine means we need to back up and look at it from a broader point of view
    \item Talka bout symmetric key encryption in general
    \item Talk about trapdoor functions


    \item Now finally we introduce ECC. This is part 1
    \item We introduce elliptic curve graph (weierstrass something something)
    \item Make them do the scalar addtion and multiplication.

    \item Now move on to part 2. Introduce congruency classes./
    \item make them do the same thing over gaussian field.

    \item idea, but hard. can make them actually do ecc and fill in the gaps using this. Problem is on my end, actually implementing ecc finna take hella time.
    \item another idea; introduce it hrough diffie helman key exchange (very intuitive, see youtube video in outline) them students can see one to one relation. maybe cover evne less RSA then I originally thought.

    \item aside. make them see in practice (use google to check security of website, any good website should be using ECC like proton or cloudlfare)
    \item aside 2. if really uninspired just make them do something with RSA (maybe something small but imma do mostly new stuff)
\end{itemize}


\section*{Part 2: Enter Elliptic Curves}

After the introduction of RSA and Diffie-Hellman, researchers explored other mathematics-based cryptographic solutions looking for other algorithms beyond factoring that would serve as good Trapdoor Functions. In 1985, cryptographic algorithms were proposed based on an (then) esoteric branch of mathematics called elliptic curves.


\subsection*{The Elliptic-curve Discrete Log Function}

% hmm, interesting argument against ECC for public: https://crypto.stackexchange.com/questions/59619/is-there-a-situation-where-rsa-cannot-be-replaced-with-ecc-symmetric-algorithm



\begin{tcolorbox}
% [colback=yellow!30]
    In this part, you'll implement point addition and point multiplication on elliptic curves, as well as over finite fields.
\end{tcolorbox}

\section*{Part 3: ECC in Action: Elliptic-curve Diffie Helman}

We'll be using the tinyec package for this part.

% https://cryptobook.nakov.com/asymmetric-key-ciphers/ecdh-key-exchange-examples

\begin{tcolorbox}
% [colback=yellow!30]
    In this part, you'll fully implement an Elliptic-curve Diffie-Helman key exchange. Of course, not by yourself. We'll be using the tinyec package to handle the elliptic curve generation. Lots of the functions are implemented for you.
\end{tcolorbox}


\subsection*{Conclusion}

Put advantages, 

use fewer memory and CPU resources, important as mobile computing becomes more ubiquitous


less storage stuff, More secure

Disadvantages, choosing the right elliptic curve, NIST distrust.


Finally, neither ECC nor RSA are secure against quantum computers.

assymetric not often used since takes long time. Instead, key exchange is done using assymetric andthen symmetric is used to send the actual messages.


\section*{Submission Instructions (10 pts)}
    After you fill the appropriate functions, convert the ECC.ipynb file to a .py file and submit it on Gradescope.

    \vspace{3mm}
    \textbf{Notes}
    \begin{itemize}
        \item The autograder may not reflect your final grade on the assignment. We reserve the right to run additional tests during grading.
        \item Do not import additional packages, as your submission may not pass the test cases or manual review.
    \end{itemize}

    

\end{document}