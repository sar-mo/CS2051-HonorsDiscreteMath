\documentclass{article}
\usepackage[margin=1in]{geometry}
\usepackage{amsmath, amssymb, amsthm}
\usepackage{enumitem}

\newenvironment{solution}
{
\medskip
\par
\color{blue}
\textbf{Solution:}
}
{
\medskip
\par
}

%Highlighting
\usepackage{xcolor, soul}
\sethlcolor{lightgray}

%Cases Environment
\newlist{Cases}{enumerate}{3}
\setlist[Cases]{leftmargin = .25in, label = {Case \arabic*.}, topsep = 0.01in, itemsep = 0.04in, itemindent = .5in, parsep = 0in}

%Formatting and Spacing
\setitemize[1]{noitemsep, parsep = 5pt, topsep = 5pt}
%\setenumerate[1]{label = (\alph*), parsep = 1pt, topsep = 5pt}
\setlength\parindent{0pt}
\linespread{1.15}

%Formatting and Spacing
\usepackage{enumitem}
\setitemize[1]{noitemsep, parsep = 5pt, topsep = 5pt}
%\setenumerate[1]{label = (\alph*), parsep = 1pt, topsep = 5pt}
\setlength\parindent{0pt}
\linespread{1.1}

% title
\title{\vspace{-1cm}CS 2051: Honors Discrete Mathematics \\Spring 2023 Homework 4 Supplement}
\author{Sean Peng\footnote{Solutions were published with the permission of the student.}}
\date{}

\begin{document}

\maketitle
\begin{enumerate}

\item The purpose of this problem is to show how the power set $\mathcal{P}(S)$ of a given set $S$, has always a different cardinality. We have a formula for the case that $S$ is finite, but it is less obvious for the infinite case. To fix ideas, we focus on the case $S=\mathbb{N}$.\\

Show that there does not exist an onto function between $\mathbb{N}$ and its power set, $\mathcal{P}(\mathbb{N})$.\\

 {\it Hint: proceed by contradiction assuming there is such function $f:\mathbb{N} \rightarrow \mathcal{P}(\mathbb{N})$. Let 
 
 \[
 T=\{n\in \mathbb{N}|~n\notin f(n) \}.
 \]

Since $f$ is onto, there exist $t\in \mathbb{N}$ such that $f(t)=T$. Argue a contradiction by looking at $t\in T$ and $t\notin T$.}
\begin{solution}

    \textbf{Proof:} I proceed with a proof by contradiction. Assume $f:\mathbb{N} \rightarrow \mathcal{P}(\mathbb{N})$ is onto. Then, since $T \subseteq \mathbb{N}$ and $T \in \mathcal{P}(\mathbb{N})$, there exists $t \in \mathbb{N}$ such that $f(t)=T$.
    
    If $t \in T$, then by the definition of $T$, $t \notin f(t)$. Since $t \in T$ and $t \notin f(t)$, $T \neq f(t)$.

    If $t \notin T$, then by the definition of $T$, $t \in f(t)$. Since $t \notin T$ and $t \in f(t)$, $T \neq f(t)$.

    We have shown that for all $t \in \mathbb{N}$, $f(t) \neq T$. Applying De Morgan's law for quantifiers, there does not exist $t \in \mathbb{N}$ such that $f(t)=T$. This contradicts with our assumption that $f$ is onto and that there exists such a $t$. Therefore, there does not exists an onto function between $\mathbb{N}$ and $\mathcal{P}(\mathbb{N})$.
\end{solution}

\item The {\tt Brito-Caribbean-Royal Grand Hotel} has a countable infinite number of rooms, each occupied by a guest, due to its popular demand.

\begin{enumerate}
    \item  How can we accommodate a new guest arriving at the fully occupied hotel without removing any of the current guests?
     \begin{solution}
        First, number the rooms with positive integers, starting from 1. Then, we move every guest from their original room $n$ to room $n + 1$. Now room 1 is empty, and the new guest can stay there.
    \end{solution}
    \item Show that a finite group of guests arriving at the Grand Hotel can be given rooms without evicting any current guests.
     \begin{solution}
        First, number the rooms with positive integers, starting from 1. Let the finite number of guests arriving at the Grand Hotel be $k$. Then, we move every current guest from their original room $n$ to room $n + k$. Now the $k$ new guests can move into the rooms numbered from 1 to $k$.
    \end{solution}
    \item Brito, the owner of the hotel, decided to close all the even numbered rooms for maintenance. Show that all the guests can remain in the hotel.
     \begin{solution}
        First, number the rooms with positive integers, starting from 1. For all guests with room numbers larger than 1, move them from their original room $n$ to room $2n - 1$. $2n - 1 = 2k + 1$, where $k = n - 1 \in \mathbb{Z}$, therefore $2n - 1$ is odd. Since 1 is odd and $2n - 1$ is odd for all $n \in \mathbb{Z}$, all even numbered rooms are empty and can be closed for maintenance.
    \end{solution}
    \item A countable infinite number of buses, each containing a countable infinite number of guests, arrive at the hotel, show that the arriving guests can be accommodated without evicting any of the current guests.
    \begin{solution}
        Number the rooms with positive integers, starting from 1. Also number the buses and the passengers in each bus the same way. Move all current guests from their current room $n$ to room $2^n$. Then, for the $i$th passenger on the $j$th bus, place them in room $(p_j)^i$, where $p_j$ is the $j$th odd prime number. For example, the second passenger on the second bus would be in room $5^2 = 25$.
    \end{solution}
    \end{enumerate}

To earn full credit, carefully describe in each part how the hotel goes about accommodating the new guests.

\end{enumerate}

\end{document}




****************


Let $S$ be the set that contains a set x if the set x does not belong to itself, so that $S = \{x \vert x \not\in x\}.$
\begin{itemize}
    \item Show the assumption that $S$ is a member of $S$ leads to a contradiction.
    \item Show the assumption that $S$ is not a member of $S$ leads to a contradiction.
\end{itemize}