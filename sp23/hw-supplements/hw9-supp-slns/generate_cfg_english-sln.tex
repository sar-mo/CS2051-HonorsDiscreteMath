\documentclass{article}
\usepackage[margin=1in]{geometry}
\usepackage{amsmath, amssymb, amsthm}
\usepackage{enumitem}

% colored links
\usepackage{hyperref}
\hypersetup{
    colorlinks=true,
    linkcolor=blue,
    filecolor=magenta,      
    urlcolor=magenta,
    }

%Creating algorithms
\usepackage{algorithm}
\usepackage[noend]{algpseudocode}


% Inputting Python code
\usepackage[dvipsnames]{xcolor}
\definecolor{textblue}{rgb}{.2,.2,.7}
\definecolor{textred}{rgb}{0.54,0,0}
\definecolor{textgreen}{rgb}{0,0.43,0}
\usepackage{upquote}
\usepackage{listings}
\lstset{
    language=Python, 
    tabsize=4,
    basicstyle={\ttfamily},
    keywordstyle=\color{textblue},
    commentstyle=\color{textgreen},
    stringstyle=\color{textred},
    frame=none,
    columns=fullflexible,
    keepspaces=true,
    showstringspaces=false,
    xleftmargin=-15mm, % manual adjustment, figure out permanent solution
}

% Tcolorbox
\usepackage{tcolorbox}
\tcbuselibrary{skins,hooks}
\usetikzlibrary{shadows}

%Images
\usepackage{graphicx}
    \usepackage{subcaption}
    \usepackage{float}

%Creating Figures
\usepackage{tikz}
\usetikzlibrary{calc, math, matrix, graphs, positioning}
\usetikzlibrary{fit}
\usetikzlibrary{backgrounds}
\usetikzlibrary{hobby}

% the settings of tikz is used for the optimization of the graphs  
\usetikzlibrary{shapes, arrows, arrows.meta} % these are the parameters passed to the library to create the node graphs  
\tikzset{  
    auto,node distance =1.1 cm and 1.1 cm,% node distance is the distance between one node to other, where 1.5cm is the length of the edge between the nodes  
} 

% macros for cfg
\newcommand{\CONJ}{\text{CONJ}}
\newcommand{\ART}{\text{ART}}

\newcommand{\NPs}{\text{NPs}}
\newcommand{\NP}{\text{NP}}
\newcommand{\NN}{\text{NN}}

\newcommand{\VP}{\text{VP}}
\newcommand{\Vi}{\text{Vi}}
\newcommand{\Vt}{\text{Vt}}

\newcommand{\ADVs}{\text{ADVs}}
\newcommand{\ADV}{\text{ADV}}
\newcommand{\ADJs}{\text{ADJs}}
\newcommand{\ADJ}{\text{ADJ}}

\newcommand{\PP}{\text{PP}}
\newcommand{\PREP}{\text{PREP}}

%Formatting and Spacing
\setitemize[1]{noitemsep, parsep = 5pt, topsep = 5pt}
\setenumerate[1]{label = (\alph*), parsep = 1pt, topsep = 5pt}
\setlength\parindent{0pt}
\linespread{1.1}

% title
\title{\vspace{-1cm}CS 2051: Honors Discrete Mathematics \\Spring 2023 \texttt{generate\_cfg\_english} Solution}
\author{Sarthak Mohanty }
\date{}

\begin{document}

\maketitle

This document aims to cover the solution to the \lstinline{generate_cfg_english} function. As a refresher, the function asked us to create a context-free grammar (CFG) capable of generating a few english sentences. To do this, we will go through each of the sentences one by one, and continually add / update our production rules to fit our requirements.

\vspace{2mm}
To simplify our CFG, we use the following abbreviations:
\begin{center}
\begin{tabular}{l@{\hskip 1cm}l}
    ART & Article \\
    CONJ & Conjunction \\
    NP & Noun Phrase \\
    ADV & Adverb \\
    ADJ & Adjective \\
    NN & Noun \\
    Vi & Intransitive Verb \\
    Vt & Transitive Verb.
\end{tabular}
\end{center}
Here, to simplify notation, we have combined four categories into one: $$\text{Noun = Singular Noun $\cup$ Plural Noun $\cup$ Proper Noun $\cup$ Pronoun}.$$ Now that we have defined our parts of speech, we can begin to develop our CFG:

\begin{enumerate}[label = \textbf{\arabic*.}]
    \item 
     \textbf{The man swims.} The sentence consists of a subject, an article, and an intransitive verb. We use the traditional notation of splitting sentences up into Noun Phrases (NP) and Verb Phrases (VP), leading to the following structure:
     $$\begin{array}{r@{ {}\rightarrow{} }lll}
        S & \NP & \VP \\
        \NP & (\ART) & \NN \\
        \VP & \Vi
    \end{array}$$
    \item \textbf{The trainer carried the dumbbells.} We now introduce transitive verbs. In this sentence, ``the dumbells" is an example of a direct object, the recipient of a transitive verb. We modify our verb phrase as follows:
    $$\begin{array}{r@{ {}\rightarrow{} }lll}
        \VP & \Vt & \NP
    \end{array}$$
    \item \textbf{Ron ate his cold, delicious protein shake} and \textbf{The very extremely tall, intelligent woman deadlifted}. I mentioned this one in the intructions, but we can add a potentially infinite number of adjectives to modify our direct object without any fault:
    $$\begin{array}{r@{ {}\rightarrow{} }lll}
        \NP & (\ART) & (\ADJs) & \NN \\
        \ADJs & \ADJ & (\ADJs)
    \end{array}$$
    \item \textbf{Reese gave her bar to Paul.} Again we must modify our verb phrase. We do so using prepositional phrases. We modify our rules as follows:
    $$\begin{array}{r@{ {}\rightarrow{} }lll}
        \VP & \Vi & (\PP) \\
        \VP & \Vt & \NP & (\PP) \\
        \PP & \PREP & \NP
    \end{array}$$
    Again the parentheses around $\PP$ means it is optional.

    \vspace{2mm} In addition, note that ``her" is a possessive noun, not an article. As a hack, we can add the rule
    $$\begin{array}{r@{ {}\rightarrow{} }lll}
        \ART & \NN
    \end{array}$$
    However, this is not too elegant, and could be improved by differentiating possessive nouns in the \lstinline{parts_of_speech} dictionary (see `Final Remarks' section below).
    \item \textbf{She ran on the treadmill quickly and enthusiastically.} Adverbs are the verbal equivalent to adjectives. However, in this case we must also include a conjunction between each adverb.
    $$\begin{array}{r@{ {}\rightarrow{} }llll}
        \VP & \Vi & (\PP) & (\ADVs_{1}) \\
        \VP & \Vt & \NP & (\PP) & (\ADVs_{1}) \\
        \ADVs_{1} & \ADV & (\ADVs_{2}) \\
        \ADVs_{2} & \CONJ & \ADVs_{1}
    \end{array}$$
    Note that adverbs can also be placed before the prepositional phrase, i.e.\ the phrase ``She quickly and enthusiastically ran on the treadmill" would be perfectly valid.
    \item \textbf{They exercised with the ab-roller and the jump-rope.} Here two noun phrases are being combined with a conjunction. We accommodate this change by allowing noun phrases to split into multiple noun phrases.
    $$\begin{array}{r@{ {}\rightarrow{} }llll}
        \NP & (\ART) & (\ADJs) & \NN & (\NPs) \\
        \NPs & \CONJ & \NP
    \end{array}$$
    \item \textbf{The motivated fellow lifted the weights, but the unmotivated fellow dropped them.} We can combine sentences using conjunctions like so:
    $$\begin{array}{r@{ {}\rightarrow{} }lll}
        S & S & \CONJ & S
    \end{array}$$
\end{enumerate}
    The final CFG is shown below:
    \begin{tcolorbox}[enhanced,colback=Dandelion!30, colframe=Dandelion!50!black,
    width=10cm,
    center,
    drop fuzzy shadow,]
    % enhanced,title=Correct Approach,
    %     colback=Dandelion!30,
    % colframe=Dandelion!50!black
    %     arc=1mm,colbacktitle=Dandelion!10,
    %     fonttitle=\bfseries,coltitle=Dandelion!50!black,
    %     attach boxed title to top center=
    %     {yshift=-2mm,yshifttext=-1mm},
    %     boxed title style={
    %     interior style={fill=none,
    %     top color=Dandelion!30!white,
    %     bottom color=Dandelion!20!white}}]
    $$\begin{array}{r@{ {}\rightarrow{} }llll}
        S & S & \CONJ & S \\
        S  & \NP & \VP \\
        \NP & (\ART) & (\ADJs) & \NN & (\NPs) \\
        \ART & \NN \\
        \ADJs & \ADJ & (\ADJs) \\
        \NPs & \CONJ & \NP \\
        \VP & \Vi & (\PP) & (\ADVs_{1}) \\
        \VP & \Vt & \NP & (\PP) & (\ADVs_{1}) \\
        \ADVs_{1} & \ADV & (\ADVs_{2}) \\
        \ADVs_{2} & \CONJ & \ADVs_{1} \\
        \PP & \PREP & \NP
    \end{array}$$
\end{tcolorbox}

Testing it with the appropriate \lstinline{parts_of_speech} and the sentences yields favorable results.

\subsection*{Final Remarks}
    If I were to do it again, I would do a few things differently.
    \begin{enumerate}[label = \arabic*.]
        \item I would split nouns into `possessive nouns' and ``common nouns", just like verbs were split up into `transitive verbs" and '`intransitive verbs". This would allow for better differentiation for parts of speech such as pronouns (``her" vs ``she").
        \item I would modify some of the test cases. The original goal was for each test case to test a different structure, but some either tested the same structure or were exceedingly difficult to implement (the latter ones were removed).
        \item Finally, I would make the exposition a bit more detailed. It's always a bit of a time crunch to make these with school and everything.
    \end{enumerate}


\end{document}