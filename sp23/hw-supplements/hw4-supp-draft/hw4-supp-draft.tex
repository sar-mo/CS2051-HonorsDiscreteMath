\documentclass{article}
\usepackage[margin=1in]{geometry}
\usepackage{amsmath, amssymb, amsthm}
\usepackage{enumitem}

%Highlighting
\usepackage{xcolor, soul}
\sethlcolor{lightgray}

%Cases Environment
\newlist{Cases}{enumerate}{3}
\setlist[Cases]{leftmargin = .25in, label = {Case \arabic*.}, topsep = 0.01in, itemsep = 0.04in, itemindent = .5in, parsep = 0in}

%Formatting and Spacing
\setitemize[1]{noitemsep, parsep = 5pt, topsep = 5pt}
\setenumerate[1]{label = (\alph*), parsep = 1pt, topsep = 5pt}
\setlength\parindent{0pt}
\linespread{1.15}

%Formatting and Spacing
\usepackage{enumitem}
\setitemize[1]{noitemsep, parsep = 5pt, topsep = 5pt}
\setenumerate[1]{label = (\alph*), parsep = 1pt, topsep = 5pt}
\setlength\parindent{0pt}
\linespread{1.1}

% title
\title{\vspace{-1cm}CS 2051: Honors Discrete Mathematics \\Spring 2023 Homework 4 Supplement}
\author{Nithya Jayakumar }
\date{}

\begin{document}

\maketitle

\section*{Cardinality}
    
    There is an easy way to test if two numbers have the same cardinality, which will become useful when we deal with infinite sets.
    
    \vspace{1.5mm}
    \textbf{Theorem.} Two sets $A$ and $B$ are said to have the same cardinality if there exists a bijection (surjection?) from $A$ to $B$. The proof of this theorem is outside the scope of this course.
    
    \vspace{1.5mm}
    \textbf{Definition}: A set is said to be \textit{countable} if it is finite or has the same cardinality as the natural numbers.

    
    \textbf{Example.} 
    Show that $\mathbb{N}$ to $A = \{x \in \mathbb{Q}: x > 0\}$ is a bijection.
    
\subsection*{Cantor's Diagonal Lemma}
    Let $f$ be a function from $\mathbb{N}$ to $(0, 1)$. Prove that there exists $y \in (0, 1)$ such that $y$ does not belong to the range of $f$. (in other words, prove the set of real numbers is not countable.)

\subsection*{Solution}
    We are given a function $f: \mathbb{N} \rightarrow (0, 1)$. We wish to find a number $y \in (0, 1)$ such that $$y \notin \{\text{$f(1)$, $f(2)$, $f(3)$, $f(4)$, \dots}\}.$$ For each $n \in \mathbb{N}$ and each $k \in \mathbb{N}$, let $x_{nk}$ be the $k$-th digit in the \textit{standard} decimal expansion of $f(n)$. Then 
    \begin{align*}
        &f(1) = 0.\text{\hl{$x_{11}$}}x_{12}x_{13}x_{14}\dots, \\
        &f(2) = 0.x_{21}\text{\hl{$x_{22}$}}x_{23}x_{24}\dots, \\
        &f(3) = 0.x_{31}x_{32}\text{\hl{$x_{33}$}}x_{34}\dots, \\
        &f(4) = 0.x_{41}x_{42}x_{43}\text{\hl{$x_{44}$}}\dots, \\
        &\text{and so on}.
    \end{align*}
    We shall define the number $y$ by defining the digits in its decimal expansion so that they are different from the ``diagonal" entries $x_{11}, x_{22}, x_{33}, x_{44}, \dots$ that are highlighted in the equations above. For each $n \in \mathbb{N}$, let
    $$y_n = 
    \begin{cases}
        5 & \text{if } x_{nn} \ne 5, \\
        4 & \text{if } x_{nn} = 5.
    \end{cases}$$
    Then for each $n \in \mathbb{N}$, $y_n \ne x_{nn}$. Now let $y$ be the number whose standard decimal expansion is $$y = 0.y_1y_2y_3y_4 \dots.$$ Then $y \in (0, 1)$. In fact, $0.444 \ldots \le y \le 0.555 \dots$. To see that $y$ is not in the range of $f$, note that for each $n \in \mathbb{N}$, $y \ne f(x)$ (because the numbers $y$ and $f(n)$ differ in their $n$-th decimal place; in other words, $y_n \ne x_{nn}$).

\end{document}