% \documentclass{article}
% \usepackage[margin=1in]{geometry}
% \usepackage{amsmath, amssymb, amsthm}
% \usepackage{enumitem}

% % colored links
% \usepackage{hyperref}
% \hypersetup{
%     colorlinks=true,
%     linkcolor=blue,
%     filecolor=magenta,      
%     urlcolor=blue,
%     }


% % Inputting Python code
% \usepackage[dvipsnames]{xcolor}
% \definecolor{textblue}{rgb}{.2,.2,.7}
% \definecolor{textred}{rgb}{0.54,0,0}
% \definecolor{textgreen}{rgb}{0,0.43,0}
% \usepackage{upquote}
% \usepackage{listings}
% \lstset{
%     language=Python, 
%     tabsize=4,
%     basicstyle={\ttfamily},
%     keywordstyle=\color{textblue},
%     commentstyle=\color{textgreen},
%     stringstyle=\color{textred},
%     frame=none,
%     columns=fullflexible,
%     keepspaces=true,
%     showstringspaces=false,
%     xleftmargin=-15mm, % manual adjustment, figure out permanent solution
% }

% %Creating algorithms
% \usepackage{algorithm}
% \usepackage[noend]{algpseudocode}

% \usepackage{tcolorbox}
% \tcbuselibrary{skins,hooks}
% \usetikzlibrary{shadows}
% \usepackage{lipsum}

% %Images
% \usepackage{graphicx}
%     \usepackage{subcaption}
%     \usepackage{float}
%     % \usepackage[labelsep=period]{caption}

% %Formatting and Spacing
% \setitemize[1]{noitemsep, parsep = 5pt, topsep = 5pt}
% \setenumerate[1]{label = (\alph*), parsep = 1pt, topsep = 5pt}
% \setlength\parindent{0pt}
% \linespread{1.1}

% % title
% \title{\vspace{-1cm}CS 2051: Honors Discrete Mathematics \\Spring 2023 Homework 5 Supplement}
% \author{Sarthak Mohanty }
% \date{}

% \begin{document}

% \maketitle

% Oultine/Ideas
% \begin{itemize}
%     \item Begin by introducing relations and their properties. Introduce partial orders. Make students check properties through code.
%     \item Make sure to establish connection between relations and functions, perhaps also explain connection between realtions and graphs, or even functions and graphs.
%     \item Another idea: Introduce dependency graph, make sutdents construct build order: to make more difficult, may be possible to not have build order
% \end{itemize}


% % \textbf{Overview}
% %     Traditionally, computer software has been written for serial computation: instructions corrresponding to a tasks are executed on one processing unit at a time.



% %   computational task consisted of many elementary operations, some of which had to be computed sequentially, while others could be computed in parallel. 

% %   In a distributed computing class, one of the first things you learn is \textbf{Amdahl’s law}, which gives a quantitative measure of the speedup $S$ -i.e., how much faster the task can be performed by using $n$ parallel processors:
% %   $$S = \frac{1}{1 - p + \frac{p}{n}}.$$ Here, $p$ is the proportion of operations that can be performed in parallel.

% %     \vspace{2mm}
% %     However, it is not typically the case that we can classify operations simply as “parallel” or “sequential.” Instead, a task might consist of several sub-tasks, some of which need to be completed before others are started. The ordering of these subtasks, as well as computing the speedup, is a much harder problem (comparable in difficulty to the \verb+SAT+ problem encountered previously).

% %     \vspace{2mm}
% %     In this supplement, you'll develop the mathematical intuition to formally represent this problem. You've learned about functions, now you'll learn about a specialized version of them known as \textbf{relations}.
    
    
% %     You'll then apply different techniques to solve and analyze different applications of this problem.





% % \section*{Relations}

% %     Recall the definition of a relation:
    
% %     \vspace{1.5mm}
% %     \textbf{Definition} a \textit{relation} is a subset of the Cartesian product $A\times B$.
    
% %     \vspace{1.5mm}
% %     We denote relations by $\mathcal{R}$. We write $a\mathcal{R} b$ to indicate that $(a,b)\in \mathcal{R}$ (i.e.: in the subset denoted by $\mathcal{R}$). When $A=B$ we said that $\mathcal{R}$ is a relation on $A$.\\
    
% %     Let $\mathcal{R}_1, \dots, \mathcal{R}_4$ be a relation on $A = \{1, 2, 3, 4\}$.
% %     \begin{itemize}
% %         \item $\mathcal{R}_1 = \{(a, b) \mid a \le b)\}$
% %         \item $\mathcal{R}_2 = \{(a, b) \mid a = b)\}$
% %         \item $\mathcal{R}_3 = \{(a, b) \mid a+b \le 2022)\}$
% %         \item $\mathcal{R}_4 = \{(a, b) \mid a \text{ divides } b\}$
% %     \end{itemize}
% %     So why relations? They are more general and allow us to study more complex sets. Elaborate. \\
    
% % \subsection*{Properties}
    
% %     \begin{itemize}
% %         \item \underline{Reflexive:} $(\forall a \in A)(a\mathcal{R}a)$
% %         \item \underline{Symmetric:} $(\forall a, b \in A)(a\mathcal{R}b \iff b\mathcal{R}a)$
% %         \item \underline{Antisymmetric:} $(\forall a, b\in A)(a\mathcal{R}b \wedge b\mathcal{R}a \rightarrow a=b)$
% %         \item \underline{Transitive:} $(\forall a, b, c\in A)(a\mathcal{R}b \wedge b\mathcal{R}c \rightarrow a\mathcal{R}c)$
% %     \end{itemize}

% %     Some examples of relations:


% % \section*{Partially Ordered Sets}

% % \subsection*{Part 1: Single Processor with Dependency}

% % \section*{Task Scheduling}

% %     % title={The title},
% %     \begin{tcolorbox}[enhanced,watermark graphics=images/circuit,
% %     watermark opacity=0.9,watermark stretch=1, watermark overzoom = 1.2]
% %         \color{white}

% %         \textbf{Your task is as follows:}
        
        
% %         \vspace{3mm}
% %         \lipsum[4]

% %         \vspace{3mm}
% %         \lipsum[3]
% %     \end{tcolorbox}

% \end{document}