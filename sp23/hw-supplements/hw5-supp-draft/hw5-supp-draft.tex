\documentclass{article}
\usepackage[margin=1in]{geometry}
\usepackage{amsmath, amssymb, amsthm}
\usepackage{enumitem}

%Highlighting
\usepackage{xcolor, soul}
\sethlcolor{lightgray}

%Cases Environment
\newlist{Cases}{enumerate}{3}
\setlist[Cases]{leftmargin = .25in, label = {Case \arabic*.}, topsep = 0.01in, itemsep = 0.04in, itemindent = .5in, parsep = 0in}

%Formatting and Spacing
\setitemize[1]{noitemsep, parsep = 5pt, topsep = 5pt}
\setenumerate[1]{label = (\alph*), parsep = 1pt, topsep = 5pt}
\setlength\parindent{0pt}
\linespread{1.15}

%Formatting and Spacing
\usepackage{enumitem}
\setitemize[1]{noitemsep, parsep = 5pt, topsep = 5pt}
\setenumerate[1]{label = (\alph*), parsep = 1pt, topsep = 5pt}
\setlength\parindent{0pt}
\linespread{1.1}

% title
\title{\vspace{-1cm}CS 2051: Honors Discrete Mathematics \\Spring 2023 Homework 5 Supplement}
\author{Sarthak Mohanty }
\date{}

\begin{document}

\maketitle

\section*{Relations}

    Recall the definition of a relation:
    
    \vspace{1.5mm}
    \textbf{Definition} a \textit{relation} is a subset of the Cartesian product $A\times B$.
    
    \vspace{1.5mm}
    We denote relations by $\mathcal{R}$. We write $a\mathcal{R} b$ to indicate that $(a,b)\in \mathcal{R}$ (i.e.: in the subset denoted by $\mathcal{R}$). When $A=B$ we said that $\mathcal{R}$ is a relation on $A$.\\
    
    Let $\mathcal{R}_1, \dots, \mathcal{R}_4$ be a relation on $A = \{1, 2, 3, 4\}$.
    \begin{itemize}
        \item $\mathcal{R}_1 = \{(a, b) \mid a \le b)\}$
        \item $\mathcal{R}_2 = \{(a, b) \mid a = b)\}$
        \item $\mathcal{R}_3 = \{(a, b) \mid a+b \le 2022)\}$
        \item $\mathcal{R}_4 = \{(a, b) \mid a \text{ divides } b\}$
    \end{itemize}
    % Create bijective (bean diagram) graph in tikz.
    
    % \begin{tikzpicture}[ele/.style={fill=black,circle,minimum width=.8pt,inner sep=1pt},every fit/.style={ellipse,draw,inner sep=-2pt}]
    %   \node[ele,label=left:$a$] (a1) at (0,4) {};    
    %   \node[ele,label=left:$b$] (a2) at (0,3) {};    
    %   \node[ele,label=left:$c$] (a3) at (0,2) {};
    %   \node[ele,label=left:$d$] (a4) at (0,1) {};
    
    %   \node[ele,,label=right:$1$] (b1) at (4,4) {};
    %   \node[ele,,label=right:$2$] (b2) at (4,3) {};
    %   \node[ele,,label=right:$3$] (b3) at (4,2) {};
    %   \node[ele,,label=right:$4$] (b4) at (4,1) {};
    
    %   \node[draw,fit= (a1) (a2) (a3) (a4),minimum width=2cm] {} ;
    %   \node[draw,fit= (b1) (b2) (b3) (b4),minimum width=2cm] {} ;  
    %   \draw[->,thick,shorten <=2pt,shorten >=2pt] (a1) -- (b4);
    %   \draw[->,thick,shorten <=2pt,shorten >=2] (a2) -- (b2);
    %   \draw[->,thick,shorten <=2pt,shorten >=2] (a3) -- (b1);
    %   \draw[->,thick,shorten <=2pt,shorten >=2] (a4) -- (b3);
    %  \end{tikzpicture}
    So why relations? They are more general and allow us to study more complex sets. Elaborate. \\
    
\subsection*{Properties}
    
    \begin{itemize}
        \item \underline{Reflexive:} $(\forall a \in A)(a\mathcal{R}a)$
        \item \underline{Symmetric:} $(\forall a, b \in A)(a\mathcal{R}b \iff b\mathcal{R}a)$
        \item \underline{Antisymmetric:} $(\forall a, b\in A)(a\mathcal{R}b \wedge b\mathcal{R}a \rightarrow a=b)$
        \item \underline{Transitive:} $(\forall a, b, c\in A)(a\mathcal{R}b \wedge b\mathcal{R}c \rightarrow a\mathcal{R}c)$
    \end{itemize}

\end{document}