\documentclass{article}
\usepackage[margin=1in]{geometry}
\usepackage{amsmath, amssymb, amsthm}
\usepackage{enumitem}

% colored links
\usepackage{hyperref}
\hypersetup{
    colorlinks=true,
    linkcolor=blue,
    filecolor=magenta,      
    urlcolor=blue,
    }



% Inputting Python code
\usepackage[dvipsnames]{xcolor}
\definecolor{textblue}{rgb}{.2,.2,.7}
\definecolor{textred}{rgb}{0.54,0,0}
\definecolor{textgreen}{rgb}{0,0.43,0}
\usepackage{upquote}
\usepackage{listings}
\lstset{
    language=Python, 
    tabsize=4,
    basicstyle={\ttfamily},
    keywordstyle=\color{textblue},
    commentstyle=\color{textgreen},
    stringstyle=\color{textred},
    frame=none,
    columns=fullflexible,
    keepspaces=true,
    showstringspaces=false,
    xleftmargin=-15mm, % manual adjustment, figure out permanent solution
}
\usepackage{tcolorbox}
\tcbuselibrary{skins,hooks}
\usetikzlibrary{shadows}
\usepackage{lipsum}

%Images
\usepackage{graphicx}
    \usepackage{subcaption}
    \usepackage{float}

%Formatting and Spacing
\setitemize[1]{noitemsep, parsep = 5pt, topsep = 5pt}
\setenumerate[1]{label = (\alph*), parsep = 1pt, topsep = 5pt}
\setlength\parindent{0pt}
\linespread{1.1}

% title
\title{\vspace{-1cm}CS 2051: Honors Discrete Mathematics \\Spring 2023 Homework 7 Supplement}
\author{Sarthak Mohanty }
\date{}

\begin{document}

\maketitle


Title: Language Building with Recursion

\section*{Overview}

% https://web.stanford.edu/class/archive/cs/cs103/cs103.1234/timeline_of_results
If you ever peruse an article about important results in discrete math, you'll often see a statement similar to ``Noam Chomsky publishes the book “Syntactic Structures."  Noam Chomsky may be many things, including a world-renowned linguist and psychologist, but it seems difficult to see his correlation to discrete math.

In 2050, you covered induction, a way to ... . In this supplement, we'll cover recursion, and delve into the intimate counterplay between induction and recursion. we'll introduce computational models, including an important one called a context-free grammar. You'll make your own called a parser,

\section*{What is Recursion}

If you already have some knowledge of recursion, you can skip this exposition.
\subsection*{Induction vs Recursion}


\begin{tcolorbox}[enhanced,interior style={top color=Dandelion!20,bottom color=Dandelion!30}]
    \textbf{In this part, you'll implement the following functions:}
    \begin{itemize}
        \item Ideas: tree parsing (sort of like implmenting eval from scratch.
    \end{itemize}
\end{tcolorbox}


\section*{Modeling Computation}
    One powerful model of computation is called a \textit{context-free-grammar}. 

    Essentially it consists of variables that can ``split" into multiple other variables. Each of these variables eventually either dissapears or turns into a string.
    
    
    We'll introduce the formal definition in a bit, but first let's just look at some examples. 


    However, CFGs are not all-powerful, as there are some languages that are not able to be generated by context-free grammars, such as $L = \{ww, w \in \Sigma^{*}\}$. The most powerful models of computation are known as \href{https://www.google.com/doodles/alan-turings-100th-birthday}{Turing machines}. Again, all of this you'll explore further in CS 4510.

\begin{tcolorbox}[enhanced,interior style={top color=Dandelion!20,bottom color=Dandelion!30}]
    \textbf{In this part, you'll implement the following functions:}
    \begin{itemize}
        \item \lstinline{generateCFG1}: Let's start easy. This CFG generates all binary strings of the form $1^{n}0^{n}$, where $n \in \mathbb{N}$.
        \item \lstinline{generateCFG2}: Ok, what about if order doesn't matter. Try to generate a CFG whose language is the binary strings with the same number of 1's and 0's!
        \item \lstinline{generateCFG3}: $\{1^{i}b^{j} : i < j < 2i \in \mathbb{N}\}$
        \item \lstinline{generateCFG4}: $\{1^{i}0^{i} : 2i \ne 3j + 1\}$
        \item \lstinline{generateCFG5}: $\{xyz \mid |x|=|y|=|z|, x\neq y \lor y \neq z \lor x \neq z\}$
        $ L = \{w\#x | w, x, \in \{0, 1\}^{*}, |w| = |x|, w \ne x^{R}$
    \end{itemize}
\end{tcolorbox}

maybes
$\{a^{n}b^{m} : m \ge n, m - n \text{ is odd}\}$

$\{a^{m}b^{i}c^{n} | m > n, 0 \le i < 3, n \ge 0\}$

$a^{m}b^{i}a^{n} : i = m + n$


% https://courses.engr.illinois.edu/cs173/fa2016/B-lecture/Lectures/Induction%20proof%20on%20Context%20Free%20Grammars.pdf

\section*{Syntactical Structures and Parsers}


\subsection*{Dynamic Programming}



\begin{tcolorbox}[enhanced,interior style={top color=Dandelion!20,bottom color=Dandelion!30}]
    \textbf{In this part, you'll implement the following functions:}
    \begin{itemize}
        \item \lstinline{test(1, 2, 3)}: Test.
        \item \lstinline{test(1, 2, 3)}: Test.
    \end{itemize}
\end{tcolorbox}


% \section*{(Optional) Context-Sensitive Grammars}

% textbook on engilsh as cfg: 

% https://www.google.com/search?q=generalized+phrase+structure+grammar&oq=Generalized+Phrase+Structure+Grammar&aqs=chrome.0.0i512l2j0i15i22i30j0i390l3.452j0j7&sourceid=chrome&ie=UTF-8


% https://www.jstor.org/stable/4178381?seq=1

example parser:

% https://i.stack.imgur.com/2gOx4.jpg

% https://courses.cs.washington.edu/courses/cse401/16wi/lectures/10-CYK-Earley-Disambig-wi16.pdf

% https://www.inf.ed.ac.uk/teaching/courses/inf2a/slides2018/inf2a_L21_slides.pdf


t
\section*{Submission Instructions (10 pts)}
    After you fill the appropriate functions, submit the following files to Gradescope and make sure you pass all test cases:
    \begin{itemize}
        \item \lstinline{recursion_sandbox.py}
        \item \lstinline{generate_CFGs.py}
        \item \lstinline{english_parser.py}
    \end{itemize}

    \vspace{3mm}
    \textbf{Notes}
    \begin{itemize}
        \item The autograder may not reflect your final grade on the assignment. We reserve the right to run additional tests during grading. \textbf{In particular, there are hidden test cases for the generate CFGs}.
        \item Do not import additional packages, as your submission may not pass the test cases or manual review.
    \end{itemize}

\section*{References}



\end{document}